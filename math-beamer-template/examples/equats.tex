\section{MATH EQUATIONS}
\begin{frame}{NOTE}
\begin{itemize}
	\item $\ allowdisplaybreaks[n]$\\
	n的值为0到4,表示分页的坚决程度,例如0表示能不分页就不分页,4表示强制分页。\\
\end{itemize}

\end{frame}

\begin{frame}{items}
	usepackage[namelimits]{amsmath} 数学公式\\
	usepackage{amssymb}             数学公式\\
	usepackage{amsfonts}            数学字体\\
	usepackage{mathrsfs}            数学花体\\
	enumerate标序
	\begin{enumerate}
	\item 经营人员M作为委托人
	\item 生产成员P作为委托人
	\item 他们作为合伙人互相监督、分享风险
	\end{enumerate}
	itemize不标序
	\begin{itemize}
	\item good1
	\item good2
	\end{itemize}
\end{frame}	

\subsection{大公式环境}
\begin{frame}{各种公式环境}
	行内公式\\
	$ a+b=c $,$ 25\% $,$ \{a,b\} $\\

	\vspace{1em}
	行间公式,tag并label
	\[x+y=z \tag{1.1}\label{1.1}\]

	\text{\$ equation \$}\\
	\text{\$\$ equation \$\$}\\
	\text{/$[ equation /$]}\\
	他们都不产生编号公式。后两种公式单独占一行,即不能嵌入正文中。\\
	用\text{\$\$}表示的公式自动居中,而\text{/$[ /$ ]}表示的公式会根据配置的全局对齐方式对齐。
\end{frame}


\begin{frame}{标准单个公式环境}
	begin\{equation\}\\
	...\\
	end\{equation\}\\
	它是最一般的公式环境,表示一个公式,默认情况下之表示一个单行的公式,但是它的功能可以通过内嵌各种其他环境进行扩展。\\
	它可以内嵌的一些关于对齐的环境将在后面介绍。
\end{frame}
	
\begin{frame}[shrink]
	\begin{tabular}{lp{5cm}p{4cm}}
		\hline Environment name &Description &Notes\\
		\hline eqnarray and eqnarray*	&Similar to align and align*	&Not recommended since spacing is inconsistent\\
		\hline multline and multline*	&First line left aligned, last line right aligned	&Equation number aligned vertically with first line and not centered as with other environments.\\
		\hline gather and gather*	&Consecutive equations without alignment&\\	 
		\hline flalign and flalign*	&Similar to {\color{red}{align}}, but left aligns first equation column, and right aligns last column&\\ %空白列也必须加"&"
		\hline alignat and alignat*	&Takes an argument specifying number of columns. Allows to control explicitly the horizontal space between equations	&You can calculate the number of columns by counting \& characters in a line and adding 1\\
		\hline
	\end{tabular}
	表格引用自:\url{http://en.wikibooks.org/wiki/LaTeX/Advanced\_Mathematics}
	其中除了eqnarray是内置的以外,其他的都需要amsmath包支持。
	\end{frame}


\begin{frame}{align,多个公式}
	居中列两个公式,并且加*不自动编号\\
	与表格环境一样,它采用“\&”分割各个对齐单元,使用“$// $”换行。它的每行是一个公式,都会独立编号。
	\begin{align*}
		f(x) &= (x+a)(x+b) \\
			 &= x^2 + (a+b)x + ab\\
	\end{align*}
\end{frame}

\begin{frame}{eqnarray,多个公式}
	equation环境,自动编号(1)(2)(.)
	\begin{eqnarray}
	\alpha+\beta &=\gamma\\
	\varepsilon+\zeta+\eta &=\theta%这里加//会产生三行编号
	\end{eqnarray}
\end{frame}

\begin{frame}{gather,数学推导}
	它是最简单的多行公式环境,自己不提供任何对齐。其中的各行公式按照全局方式分别对齐。\\
	在设置了全局左对齐之后,因为不存在内部各个公式之间对排版的干扰,这种环境非常适合写数学推导或者证明。\\
	\begin{gather*}
	E(X)=\lambda	\qquad	D(X)=\lambda	\\
	E(\bar{X})=\lambda	\\
	D(\bar{X})=\frac{\lambda}{n}	\\
	E(S^2)=\frac{n-1}{n}\lambda	\\
	\end{gather*}
\end{frame}

\begin{frame}{alignat,多列公式对齐}
	它接收一个参数用来指定根据哪一列对齐。
	\begin{alignat}{2}
	 \sigma_1 &= x + y  &\quad \sigma_2 &= \frac{x}{y} \\  
	 \sigma_1' &= \frac{\partial x + y}{\partial x} & \sigma_2'
		&= \frac{\partial \frac{x}{y}}{\partial x}
	\end{alignat}

\end{frame}

\subsection{用于内嵌的对齐环境}
\begin{frame}
	这些环境无法独立构成一个数学环境,必须要嵌入在其他环境内部。
	\begin{tabular}{lp{7cm}}
		\hline Math environment name	&Description\\
		\hline gathered	&Allows to gather few equations to be set under each other and assigned a single equation number\\
		split	&Similar to align*, but used inside another displayed mathematics environment\\
		aligned	&Similar to align, to be used inside another mathematics environment.\\
		alignedat	&Similar to alignat, and just as it, takes an additional argument specifying number of columns of equations to set.\\
		
		\hline
	\end{tabular}

	\vspace{1em}
	表格引用自:\url{http://en.wikibooks.org/wiki/LaTeX/Advanced_Mathematics}.\\
	这些环境都需要amsmath包支持。
\end{frame}

\begin{frame}{内嵌split}
	它用于将一个公式拆分成多行,但是它整体还只是一个公式。\\可以用在equation,\$\$,\text{\ $[ \ $]}三个环境中,相当于align
	\begin{equation}
	 \begin{split}
	 (a + b)^4
	   &= (a + b)^2 (a + b)^2      \\
	   &= (a^2 + 2ab + b^2)	    	
		  (a^2 + 2ab + b^2)        \\
	   &= a^4 + 4a^3b + 6a^2b^2 + 4ab^3 + b^4\\
	 \end{split}
	\end{equation}
\end{frame}

\begin{frame}{内嵌aligned,多行对齐公式}
	\begin{equation}
		\left.\begin{aligned}
			   B'&=-\partial \times E,\\
			   E'&=\partial \times B - 4\pi j,
			  \end{aligned}
		\right\}
		\qquad \text{Maxwell's equations}
	   \end{equation}
	   left和right后加一个括号的表示用于自动调整各种括号的大小,必须配对使用。公式中的left. 是一个虚的left,目的是为了和right\}配对。
\end{frame}

\begin{frame}{内嵌array}
	用\&\&分开两列公式,rcl表示三列,分别右中左对齐
	\begin{equation}
		\omega_j=\left\{
		\begin{array}{rcl}
		\omega_j^s && b_i=0\\
		\omega_j^s+F_j(a_j^b) && b_i>0\tag{2.4}\label{2.4}	
		\end{array} \right.
		\end{equation}
\end{frame}