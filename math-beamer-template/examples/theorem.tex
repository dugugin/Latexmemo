% -*- coding: utf-8 -*-
\section{Theorem}
% --------------------------------------------------- Slide --
\subsection{Theorem Code}
\label{theoremCode}
\begin{frame}{Theorem}
  There is also a group of blocks that are especially useful for presenting mathematics. For example the ‘theorem’ environment, the ‘corollary’ environment and the ‘proof’ environment.
  \begin{semiverbatim}
    \\begin\{theorem\}[Pythagoras] \newline
      $ a^2 + b^2 = c^2$ \newline
    \\end\{theorem\} \newline
    \\begin\{corollary\} \newline
      $ x + y = y + x  $ \newline
    \\end\{corollary\} \newline
    \\begin\{proof\} \newline
      $\omega +\phi = \epsilon $ \newline
    \\end\{proof\}
  \end{semiverbatim}
\end{frame}

% --------------------------------------------------- Slide --
\subsection{Theorem Blocks}
\label{theoremBlocks}
\begin{frame}{Theorem Blocks}
  \begin{theorem}[Pythagoras]
    $ a^2 + b^2 = c^2$
  \end{theorem}
  \begin{corollary}
    $ x + y = y + x  $
  \end{corollary}
  \begin{proof}
    $\omega +\phi = \epsilon $
  \end{proof}
  As definition, you can also use the custom one ``thm/cor'' which is defined in ``slides/usrdef.tex''.
\end{frame}
